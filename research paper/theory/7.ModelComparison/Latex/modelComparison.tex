As the source of the traces to be analyzed come from the execution of a query in UPPAAL,
%
%As there exists an original model with its recreated version, we can now retrieve traces from both of them and compare their values.
%
we us the same query to also obtain traces from the recreated model. 
%
This ensures that the same number of traces are extracted from the two models and under the same conditions. 
%
It is explained in the following section how the comparison of the traces from the two models is performed, and how the compatibility of their functionalities is determined. 

\section{The Trace-Comparison Algorithm}
%We need to add the reference of the name of the variable
For simplicity, we assume that the query was already executed in the recreated model and that the traces were stored in the variable $simResults$. 
%
We also assume that the original trace from the first simulation is already stored in a list of simulations called $originalResults$. 
%
The filtering process of the data from the traces of a recreated model is the same as the one explained in Chapter \ref{Model Recreation}. 
%
The only difference is that when a numerical trace is found in $simResults$, we now proceed and compare its values against the traces from  $originalResults$.

\input{./algorithms/UPPAAL/traceComparisonAlgorithm.tex}

The comparison of the data of a trace from a recreated model consists on searching through all of the original traces and to compare the values of a variable at all time units of the simulation. 
%
As all traces have a tree map structure, it is very simple to retrieve the value of a variable at an exact time unit, by using the $getValue()$ method. 
%
It only requires a key that represents the $x$ value of a trace (or time unit of the simulation) in order to return the value of the variable that was analyzed.
%
The $getValue()$ method is used to obtain a value of a variable from $simResults$, based on the key of every trace stored in $originalResults$.
%
If a value from $simResults$ is obtained, then the original value from $originalResults$ is retrieved by using the same key.
%
Whenever the two values differ significantly, it means that a functionality of the recreated model does not match the functionality of the original model in a certain time unit.  
%
And if a value cannot be retrieved from $simResults$, it means that the recreated model was not able to identify a behavior from a certain time unit of the original model.  
%
%The limitations are that the traces are compared against each other at every time unit of the simulation. 
%
This approach may give negative results when comparing traces of non-deterministic systems. 
%
It might be the case that two models output different traces with the same non-deterministic functionality. 
%
In this scenario, it is most likely that two models are distinguished as incompatible even if they are not. 
%

